\chapter{Conclusion}
\label{cha:conclusion}

This thesis has explored the concept of allostery in proteins.
The in-silico methods AlloPred and ExProSE have been described along with the application of allosteric prediction approaches to CDK2.
For many years papers have pointed to the immense potential of allostery for both understanding and drugging proteins.
Yet they regularly contain the qualification that a unified framework of allostery remains `elusive', and approved allosteric drugs remain rare more than 50 years after the first descriptions of allostery.
In order to unlock the dormant potential of allostery, predictive methods need to be as established and robust as those in other areas of bioinformatics.
When allosteric prediction is as effective as prediction of secondary structure or disordered regions, the power of allostery will be truly revealed.
However the recent emergence of methods such as those presented here means the future of allosteric prediction looks bright.
In an analogous way to allostery itself, it is hoped that the effects of the field are propagated to all areas of structural biology.
