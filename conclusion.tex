\chapter{Conclusion}
\label{cha:conclusion}

This thesis has explored the concept of allostery in proteins.
The AlloPred computational approach uses normal modes and machines learning to predict allosteric pockets on proteins.
The ExProSE computational procedure uses two structures of a protein to generate ensembles that span conformational space and can be perturbed to predict and explore allostery.
Allosteric prediction methods and virtual screening were used to predict modulators for a potential allosteric site on a protein kinase important for cell cycle control, CDK2.
These modulators were tested experimentally.

Themes of this work have included the benefits and drawbacks of NMA, MD and distance geometry methods; the separation of finding good binding sites in general with potential allosteric sites; and the difficulty of comparing allosteric prediction methods.
Ultimately, the variety of allosteric mechanisms and frameworks for studying allostery makes prediction challenging and indicates that more experimental and computational work is required in this important area.

For many years papers have pointed to the immense potential of allostery for both understanding and drugging proteins.
Yet they regularly contain the qualification that a unified framework of allostery remains `elusive', and approved allosteric drugs remain rare more than 50 years after the first descriptions of allostery.
In order to unlock the dormant potential of allostery, predictive methods need to be as established and robust as those in other areas of bioinformatics.
When allosteric prediction is as effective as prediction of secondary structure or disordered regions, the power of allostery will be truly revealed.
However the recent emergence of methods such as those presented here means the future of allosteric prediction looks bright.
In an analogous way to allostery itself, it is hoped that the effects of exploring allostery will propagate to all areas of structural biology.
